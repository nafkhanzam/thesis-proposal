\section{Masalah Penelitian}

Teknik \textit{cross-lingual} \amrparsing~terbaru untuk Bahasa Indonesia adalah teknik oleh \textcite{putra2022} yang menggunakan teknik stog oleh \citek{zhang2019}.
Sudah banyak teknik-teknik lain yang memiliki kinerja \amrparsing~yang lebih baik, seperti AMRBART.
Dataset yang digunakan oleh \textcite{putra2022} juga belum menggunakan dataset terbaru, yakni AMR 3.0, yang memiliki kalimat dan aturan anotasi lebih banyak dari AMR 2.0.

Teknik \amrparsing~yang bukan termasuk teknik \textit{ensemble} terbaik saat ini adalah teknik AMRBART yang berbasis \textit{language model} BART oleh \textcite{lewis2020}. Terdapat \textit{language model} bernama PALM oleh \textcite{bi2020} yang kinerjanya lebih baik dibandingkan BART untuk \textit{task} \textit{question-answering}, peringkasan teks, \textit{question generation}, dan \textit{conversational response generation}.
\textit{Language model} PALM belum pernah digunakan untuk \amrparsing~berbahasa Inggris maupun Indonesia dan berpotensi untuk menghasilkan kinerja yang lebih baik dibandingkan teknik yang digunakan sebelumnya.

% Thesis Problem(?)
