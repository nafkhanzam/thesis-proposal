\section{Latar Belakang}

% NLP
Teks yang dapat dipahami oleh manusia tidak dapat dipahami secara langsung oleh model \gls{NLP} sehingga diperlukan sebuah bentuk representasi lain.
Ada beberapa teknik dalam membuat representasi dari sebuah teks, seperti \ti{bag of words}, \ti{n-gram}, \ti{word embedding}, \gls{SRL}, dan \AMR{}.
Dari representasi-representasi tersebut, dapat dilakukan berbagai macam \ti{task} seperti peringkasan teks, klasifikasi sentimen, mesin translasi, \ti{question-answering}, deteksi parafrasa, dan lain-lain.

% AMR
\AMR{} merupakan salah satu representasi yang memperhatikan semantik pada tingkat kalimat dalam bentuk struktur data graf berarah yang mempunyai akar \citek{banarescu2013}.
\AMR{} awalnya didesain untuk merepresentasikan kalimat berbahasa Inggris sehingga bentuk graf \AMR{} juga dituliskan dengan Bahasa Inggris.
Sebuah teks yang mengandung banyak kalimat dapat direpresentasikan menjadi beberapa graf \AMR{}, dengan setiap graf merepresentasikan setiap kalimat.
Dalam membuat sebuah \AMR{} dari sebuah kalimat, perlu dilakukan sebuah proses yang dinamakan \amrparsing{}.

% AMR Parsing Techniques
\amrparsing{} dapat dikategorikan menjadi dua pendekatan, yaitu dua tahap \ti{parsing} dan satu tahap \ti{parsing} \citek{cai2020}.
Pada pendekatan dua tahap \ti{parsing} digunakan desain \ti{pipeline} untuk identifikasi konsep dan prediksi relasi.
Pada pendekatan satu tahap \ti{parsing} dikategorikan menjadi tiga jenis, yaitu \ti{parsing} berbasis transisi, \ti{parsing} berbasis \gls{seq2seq}, dan \ti{parsing} berbasis graf.
Pendekatan satu tahap jenis \ti{parsing} berbasis transisi dilakukan dengan memproses kalimat dari kiri ke kanan dan membangun grafik secara bertahap dengan secara bergantian memasukkan simpul atau sisi baru.
Pendekatan satu tahap jenis \ti{parsing} berbasis \gls{seq2seq} dengan melihat \ti{parsing} sebagai transduksi urutan linear ke urutan linear juga dengan memanfaatkan linearisasi grafik AMR.
Pendekatan satu tahap jenis \ti{parsing} berbasis graf di mana setiap langkah waktu, simpul baru beserta koneksinya ke simpul yang ada diputuskan bersama secara berurutan maupun paralel.

% Metrics and Dataset
Dalam mengukur kelayakan hasil graf \AMR{} yang dihasilkan dari suatu teknik, digunakan metrik \SMATCH{} \citek{cai2013}.
SMATCH mengukur derajat \ti{overlap} antara dua struktur fitur semantik graf \AMR{}.
Teknik-teknik \amrparsing{} tersebut diukur kelayakannya menggunakan dataset pasangan teks berbahasa Inggris dengan graf AMR-nya.
% Dataset yang pertama kali dikeluarkan adalah AMR 1.0 (LDC2014T12) yang mengandung sebanyak 13.051 kalimat.
% Dataset tersebut kemudian dikembangkan lagi dengan tambahan kalimat dan aturan anotasi menjadi AMR 2.0 (LDC2017T10) sebanyak 39.260 kalimat dan AMR 3.0 (LDC2020T02) sebanyak 59.255 kalimat.

% ID-ID Conventional AMR Parsing
Pada \AMR{} berbahasa Indonesia, telah dikembangkan untuk \ti{parsing} berbasis aturan \citek{severina2019}, berbasis \ti{dependency parser} \citek{ilmy2020}, dan berbasis \crosslingual{} \citek{putra2022}.
Teknik \amrparsing{} berbasis aturan dan \ti{dependency parser} masih terbatas karena kurangnya panduan anotasi dan PropBank untuk adaptasi \AMR{} dalam Bahasa Indonesia.
Dataset yang digunakan sebagai data pelatihan teknik tersebut terdiri dari 1,130 pasang kalimat dan AMR berbahasa Indonesia.
Jumlah dataset tersebut masih sangat kurang untuk menghasilkan model yang berkualitas tinggi jika dibandingkan dengan jumlah dataset berbahasa Inggris yang memiliki 59.255 pasang.
Anotasi untuk \AMR{} Bahasa Indonesia hanya terbatas pada 6 relasi saja, yaitu \code{:ARG0}, \code{:ARG1}, \code{:name}, \code{:time}, \code{:location}, \code{:mod} \citek{ilmy2020}.
Konsep yang membutuhkan argumen relasi lebih banyak yang tidak tersedia akan diabaikan.
Sehingga AMR Bahasa Indonesia tidak dapat merepresentasikan kalimat yang lebih panjang dan kompleks.

% XL-AMR
\amrparsing{} awalnya digunakan untuk mengubah dari kalimat berbahasa Inggris menjadi sebuah graf \AMR{} berbahasa Inggris.
\AMR{} dalam bahasa selain Bahasa Inggris memiliki banyak limitasi karena \AMR{} berbahasa Inggris sudah dikembangkan lebih dahulu dan memiliki anotasi konsep dan relasi lengkap yang tidak dimiliki \AMR{} bahasa lain.
Ada beberapa teknik yang dapat membaca kalimat bahasa lain menjadi graf \AMR{} berbahasa Inggris untuk merepresentasikan kalimat dari bahasa selain Bahasa Inggris.
\textcite{damonte2018} mengusulkan teknik \amrparsing{} lintas bahasa, yaitu teknik mengubah kalimat dari bahasa selain Bahasa Inggris menjadi \AMR{} berbahasa Inggris.
Dalam teknik tersebut, dibutuhkan dataset pasangan teks berbahasa selain Bahasa Inggris yang dituju dengan graf \AMR{}-nya.
Model \gls{XL-AMR} memiliki kinerja SMATCH 53.0 untuk Bahasa Cina, 53.0 untuk Bahasa Jernam, 58.1 untuk Bahasa Italia, dan 58.0 untuk Bahasa Spanyol.

% XL-AMR Indonesia
Pada Bahasa Indonesia, beberapa teknik \amrparsing{} untuk teks berbahasa Indonesia juga dikembangkan.
Beberapa \amrparsing{} Bahasa Indonesia yang telah dikembangkan adalah \amrparsing{} berbasis aturan \citek{severina2019}, berbasis \ti{dependency parser} \citek{ilmy2020}, dan berbasis lintas bahasa \citek{putra2022}.
Teknik \amrparsing{} oleh \textcite{putra2022} menggunakan teknik \ti{training} \gls{XL-AMR} \citek{blloshmi2020} dengan model Sequence-to-Graph Transduction \citek{zhang2019} untuk melakukan \amrparsing{} dari kalimat berbahasa Indonesia menjadi graf \AMR{} berbahasa Inggris.
\amrparsing{} dengan Sequence-to-Graph Transduction \citek{zhang2019} menggunakan pendekatan dua tahap \ti{parsing}, yaitu tahap identifikasi konsep dan tahap prediksi relasi.
Teknik \amrparsing{} lintas bahasa dapat memanfaatkan dataset silver yang dapat dibuat dari dataset AMR dan korpus paralel.
Korpus paralel pasangan Bahasa Inggris dan Bahasa Indonesia seperti PANL-BPPT \citek{bppt2009} dan IWSLT2017 \citek{cettolo2017} dapat digunakan untuk \amrparsing{} untuk kalimat berbahasa Indonesia.

\textcite{putra2022} menggunakan dataset AMR 2.0 dan korpus paralel PANL-BPPT sebagai data latihnya.
Kalimat Bahasa Inggris dari dataset AMR 2.0 ditranslasi menjadi Bahasa Indonesia dan kalimat Bahasa Inggris dari korpus paralel PANL-BPPT dilakukan \ti{parsing} ke graf \AMR{}.
\textcite{putra2022} mengevaluasi kualitas dataset dengan menilai kedekatan kalimat hasil translasi dari AMR 2.0 menggunakan \cossim{}.
Namun, kualitas dataset hasil \amrparsing{} dari korpus paralel tidak dievaluasi.
Model terbaik dari penelitian \textcite{putra2022} menghasilkan kinerja SMATCH 51.0.
Kinerja tersebut masih kalah dengan \ti{baseline}-nya yang menggunakan teknik \ti{translate-and-parse} dengan kinerja SMATCH 62.5.
Hal ini diduga karena jumlah dataset yang dihasilkan masih belum maksimal dan dataset yang dihasilkan belum menggunakan model AMR parser dengan kualitas yang lebih baik.

\textcite{lee2022} memperkenalkan teknik augmentasi data dalam menghasilkan data latih untuk model \amrparsing{} lintas bahasa.
Karena keterbatasan jumlah dataset yang dipunya untuk melatih model \amrparsing{} lintas bahasa, teknik ini menghasilkan lebih banyak pasangan graf \AMR{} dan kalimat dengan bahasa yang dituju.
Teknik tersebut meningkatkan kinerja SMATCH pada model XL-AMR \citek{blloshmi2020} menjadi 63.0 untuk Bahasa Cina, 73.7 untuk Bahasa Jernam, 76.1 untuk Bahasa Italia, dan 77.1 untuk Bahasa Spanyol.

% PALM
Teknik \amrparsing{} terbaik saat ini adalah teknik \gls{AMRBART} dengan kinerja SMATCH 85.4 pada dataset \AMR{} 2.0 dan 84.2 pada dataset \AMR{} 3.0 \citek{bai2022}.
\gls{AMRBART} dibangun dari transformer-based \ti{language model} \gls{BART}.
\gls{PALM} merupakan sebuah teknik pre-training yang dapat menghasilkan kualitas \ti{language model} yang cukup baik untuk \ti{task} generasi teks.
\gls{PALM} dapat menghasilkan kinerja yang lebih baik dari \gls{BART} untuk \ti{task} \ti{question answering}, peringkasan teks, generasi pertanyaan, dan generasi respon.
\gls{PALM} belum pernah digunakan untuk \ti{task} \amrparsing{} dan membuka kemungkinan untuk menghasilkan kualitas yang lebih baik.
