\section{Metodologi}

Berikut adalah tahapan atau metodologi yang dilakukan dalam proses penulisan tesis ini.

\begin{enumerate}
  \item Analisis Masalah

  Tahap awal dilakukan analisis permasalahan dalam melakukan \amrparsing{} dalam Bahasa Indonesia.
  Hal ini dilakukan dengan mendalami materi mengenai teknik-teknik \amrparsing{} yang ada sebelumnya dan membandingkan hasil kinerjanya.

  \item Perancangan Solusi dan Implementasi

  Pada tahap ini dilakukan perancangan dan implementasi berdasarkan hasil analisis teknik-teknik \amrparsing{} sebelumnya.
  Dirancang sebuah solusi dengan menggunakan beberapa gabungan teknik dan model tersebut.
  Lalu dilakukan implementasi dari rancangan tersebut dengan melakukan \textit{training} model terhadap dataset yang tersedia.

  \item Pengujian dan Evaluasi

  Hasil implementasi sebelumnya diuji dan dievaluasi secara kuantitatif dengan menggunakan metrik tertentu untuk dibandingkan hasilnya dengan teknik sebelumnya.
  Hasil \amrparsing{} juga dievaluasi secara kualitatif dengan analisis divergensi translasi.
  Evaluasi secara kuantitatif dan kualitatif tersebut digunakan untuk kesimpulan kelayakan model \amrparsing{} \textit{cross-lingual} untuk kalimat berbahasa Indonesia.

\end{enumerate}
