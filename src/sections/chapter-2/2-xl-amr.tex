\section{\Glsfirst{XL-AMR}}

% \amrparsing{} merupakan proses mengubah kalimat dari natural language ke representasi \gls{AMR} yang berkorespondensi \citek{banarescu2013}.
Sebuah \gls{AMR} merupakan graf dengan node yang merepresentasikan konsep dari kalimat dan edge yang merepresentasikan hubungan semantik antar kalimat.
Dataset \gls{AMR} yang cukup besar untuk di-\textit{train} berisikan pasangan kalimat berbahasa Inggris dengan graf \gls{AMR}.
Properti \textit{cross-lingual} dari \gls{AMR} di berbagai bahasa merupakan subjek yang sering dibahas.
\textcite{banarescu2013} menyatakan bahwa \gls{AMR} bukanlah representasi lintas bahasa dan dapat dikategorikan menjadi jenis \gls{AMR} yang berbeda untuk anotasi dari bahasa yang berbeda.
\cref{fig:xl-amr-example} merupakan contoh graf \gls{AMR} dengan pasangan kalimat Bahasa Inggris dan Italia.

\fig{xl-amr-example}
  {sections/chapter-2/2-amr-example.png}
  {Contoh graf \gls{AMR} dengan pasangan kalimat Bahasa Inggris dan Italia \citek{damonte2018}.}

Tujuan dari \gls{AMR} adalah untuk mengabstraksikan realisasi sintaktik dari kalimat asli serambi mempertahankan makna yang tersirat.
Sebagai konsekuensi, perbedaan frasa yang berbeda dari satu kalimat diharapkan untuk memberikan representasi \gls{AMR} yang identik.
Namun, hal ini tidak selalu berlaku untuk lintas bahasa.
Dua kalimat yang kalimat yang mengekspresikan makna yang sama dalam dua yang sama dalam dua bahasa yang berbeda tidak menjamin untuk menghasilkan struktur \gls{AMR} yang identik \citek{xue2014}.
Dalam mengatasi permasalahan ini, \textcite{damonte2018} mengusulkan dua metode yang berbeda.

\begin{enumerate}
  \item Proyeksi anotasi.
  Setiap simpul pada graf \gls{AMR} dapat dipasangkan dengan kata pada kalimat berbahasa Inggris.
  Dengan asumsi bahasa lain memiliki struktur yang mirip dengan Bahasa Inggris, maka dapat dilakukan pemasangan terhadap bahasa lain tersebut.

  % \gls{AMR} tidak didasarkan pada kalimat input, oleh karena itu tidak perlu mengubah anotasi \gls{AMR} saat memproyeksikan ke bahasa lain.
  % Metode ini menganggap label bahasa Inggris untuk simpul-simpul graf sebagai label dari bahasa independen, yang secara kebetulan terlihat mirip dengan bahasa Inggris.
  % Namun, untuk melatih parser \gls{AMR} yang canggih, metode ini juga perlu memproyeksikan keselarasan antara node \gls{AMR} dan kata-kata dalam kalimat (selanjutnya disebut \gls{AMR}).
  % Metode ini menggunakan penjajaran kata, sama halnya dengan pekerjaan proyeksi anotasi lainnya, untuk memproyeksikan \gls{AMR} ke bahasa target.
  % Pendekatan kami bergantung pada asumsi mendasar yang kami buat: jika kata sumber disejajarkan dengan kata target dan \gls{AMR} disejajarkan dengan simpul \gls{AMR}, maka kata target juga disejajarkan dengan simpul \gls{AMR} tersebut.

  \item Mesin translasi.
  Pada metode ini digunakan sebuah mesin translasi untuk menerjemahkan kalimat dengan bahasa lain menjadi Bahasa Inggris.
  Hasil kalimat berbahasa Inggris tersebut kemudian diubah menjadi graf \gls{AMR}.
  Kualitas dari hasil graf \gls{AMR} tersebut bergantung pada kualitas mesin translasi yang digunakan.

  % Pada metode ini, dipanggil sistem MT untuk menerjemahkan kalimat input ke dalam bahasa Inggris sehingga dapat digunakan parser berbahasa Inggris yang tersedia untuk memperoleh grafik \gls{AMR}.
  % Tentu saja, kualitas dari grafik keluaran tergantung pada kualitas terjemahan kualitas terjemahan.
  % Jika terjemahan otomatis dekat dengan terjemahan referensi, maka grafik \gls{AMR} yang grafik \gls{AMR} yang diprediksi akan mendekati grafik \gls{AMR} referensi.
  % Oleh karena itu terbukti bahwa metode ini tidak informatif dalam hal sifat lintas bahasa dari \gls{AMR}.
  % Namun, kesederhanaannya menjadikannya solusi teknik yang menarik untuk mengurai bahasa-bahasa lain.
\end{enumerate}

\textcite{blloshmi2020} mengusulkan konfigurasi dalam melakukan training untuk \amrparsing{} \textit{cross-lingual}.
Konfigurasi tersebut adalah sebagai berikut.
\begin{enumerate}
  \item Zero-shot.
  Model dilatih pada kalimat berbahasa Inggris saja dengan mengandalkan fitur \textit{multilingual}, dan dievaluasi pada bahasa yang dituju.

  \item Language-specific.
  Model dilatih pada kalimat berbahasa bahasa yang dituju, misal Bahasa Indonesia, dan dievaluasi pada bahasa yang dituju tersebut.

  \item Bilingual.
  Model dilatih pada kalimat berbahasa Inggris dan bahasa yang dituju. dan dievaluasi pada bahasa yang dituju tersebut.

  \item Multilingual.
  Model dilatih pada kalimat-kalimat dengan berbagai macam bahasa, dan dievaluasi pada bahasa-bahasa yang dituju tersebut.
\end{enumerate}
