\section{\Glsfirst{XL-AMR}}

\gls{AMR} parsing merupakan proses mengubah kalimat dari natural language ke representasi \gls{AMR} yang berkorespondensi \citek{banarescu2013}.
Sebuah \gls{AMR} merupakan graf dengan node yang merepresentasikan konsep dari kalimat dan edge yang merepresentasikan hubungan semantik antar kalimat, Kebanyakan dataset \gls{AMR} yang cukup besar untuk di-train berisikan pasangan kalimat berbahasa Inggris dengan graf \gls{AMR}.
Properti cross-lingual dari \gls{AMR} di berbagai bahasa merupakan subjek yang sering dibahas.
Panduan \gls{AMR} menyatakan bahwa \gls{AMR} bukanlah interlingua \citek{banarescu2013} dan dapat dikategorikan dengan jenis yang berbeda antara anotasi dari bahasa yang berbeda.

\fig{xl-amr-example}
  {sections/chapter-2/2-amr-example.png}
  {Contoh graf \gls{AMR} dengan pasangan kalimat Bahasa Inggris dan Italia.}

Tujuan dari \gls{AMR} adalah untuk mengabstraksikan realisasi sintaktik dari kalimat asli serambi mempertahankan makna yang tersirat.
Sebagai konsekuensi, perbedaan frasa yang berbeda dari satu kalimat diharapkan untuk memberikan representasi \gls{AMR} yang identik.
Kanonikalisasi ini tidak selalu berlaku lintas bahasa.
Dua kalimat yang kalimat yang mengekspresikan makna yang sama dalam dua yang sama dalam dua bahasa yang berbeda tidak dijamin untuk menghasilkan struktur \gls{AMR} yang identik (Xue et al, 2014).
Dalam mengatasi permasalahan ini, diajukan dua metode yang berbeda.

\begin{enumerate}
  \item Proyeksi anotasi.

  \gls{AMR} tidak didasarkan pada kalimat input, oleh karena itu tidak perlu mengubah anotasi \gls{AMR} saat memproyeksikan ke bahasa lain.
  Metode ini menganggap label bahasa Inggris untuk simpul-simpul graf sebagai label dari bahasa independen, yang secara kebetulan terlihat mirip dengan bahasa Inggris.
  Namun, untuk melatih parser \gls{AMR} yang canggih, metode ini juga perlu memproyeksikan keselarasan antara node \gls{AMR} dan kata-kata dalam kalimat (selanjutnya disebut \gls{AMR}).
  Metode ini menggunakan penjajaran kata, sama halnya dengan pekerjaan proyeksi anotasi lainnya, untuk memproyeksikan \gls{AMR} ke bahasa target.
  Pendekatan kami bergantung pada asumsi mendasar yang kami buat: jika kata sumber disejajarkan dengan kata target dan \gls{AMR} disejajarkan dengan simpul \gls{AMR}, maka kata target juga disejajarkan dengan simpul \gls{AMR} tersebut.

  \item Translasi mesin.

  Pada metode ini, dipanggil sistem MT untuk menerjemahkan kalimat input ke dalam bahasa Inggris sehingga dapat digunakan parser berbahasa Inggris yang tersedia untuk memperoleh grafik \gls{AMR}.
  Tentu saja, kualitas dari grafik keluaran tergantung pada kualitas terjemahan kualitas terjemahan.
  Jika terjemahan otomatis dekat dengan terjemahan referensi, maka grafik \gls{AMR} yang grafik \gls{AMR} yang diprediksi akan mendekati grafik \gls{AMR} referensi.
  Oleh karena itu terbukti bahwa metode ini tidak informatif dalam hal sifat lintas bahasa dari \gls{AMR}.
  Namun, kesederhanaannya menjadikannya solusi teknik yang menarik untuk mengurai bahasa-bahasa lain.
\end{enumerate}