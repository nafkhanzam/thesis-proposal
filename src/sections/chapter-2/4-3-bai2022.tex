\subsection{\Glsfirst{AMRBART} \citek{bai2022}}

\Glsfirst{PLM} telah terbukti dapat melakukan \textit{task} \amrparsing{} dan generasi AMR-to-text dengan baik.
Namun, \gls{PLM} umumnya dilatih pada data tekstual, sehingga tidak optimal untuk melakukan generasi data terstruktur seperti \gls{AMR}.
\textcite{bai2022} memperbaiki permasalahan tersebut dengan menambahkan strategi \textit{pre-training} pada model untuk mengintegrasikan informasi teks dan graf \gls{AMR}.
Model ini melinearisasi graf \gls{AMR} ke sekuens sehingga baik AMR parsing dan AMR-to-text generation dapat dilakukan menggunakan model \gls{seq2seq}.
Model ini melakukan pre-training pada struktur \gls{AMR} menggunakan \gls{BART}.

% Mengikuti Kontes 2017, model ini mengadopsi algoritma Depth-First-Search (DFS) yang berhubungan dekat dengan natural language syntactic trees (Bevilacqua. 2021).

Model ini mengenalkan dua strategi \textit{self-supervised training} dalam melakukan \textit{pre-training} model \gls{BART} pada graf \gls{AMR}.
Dapat dilihat pada \cref{fig:4-3-pre-training-strategies}, strategi level \textit{denoising} simpul/sisi mendukung model untuk menangkap pengetahuan lokal mengenai simpul dan sisi.
Strategi \textit{denoising} level graf mengarahkan model untuk memprediksi sub-graf yang dapat memfasilitasi pembelajaran graf.
\begin{enumerate}
  \item \textit{Denoising} level simpul/sisi.
  Pengaplikasian fungsi noise pada simpul dan sisi AMR untuk mengkonstruksi input graf yang kotor.
  Fungsi noise ini diimplementasikan dengan \textit{masking} 15\% simpul dan 15\% sisi di setiap graf.

  \item \textit{Denoising} level sub-graf.
  \textit{Task} ini bertujuan untuk mengembalikan graf lengkap ketika diberikan sebagian dari graf.
  Metode ini menghilangkan sub-graf secara acak dari graf dan mengubahnya dengan token \textit{mask}.
\end{enumerate}

\fig[1]{4-3-pre-training-strategies}
  {sections/chapter-2/4-3-pre-training-strategies.png}
  {Ilustrasi strategi pre-training: 1) \textit{denoising} level simpul/sisi (a->b); 2) \textit{denoising} level sub-graf (c->b) \citek{bai2022}.}
