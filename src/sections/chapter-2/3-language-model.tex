\section{\ti{Language Model}}

Teks yang dapat dipahami oleh manusia tidak dapat dipahami secara langsung oleh model \gls{NLP} sehingga diperlukan sebuah bentuk representasi lain.
Ada beberapa teknik dalam membuat representasi dari sebuah teks, seperti \ti{bag of words}, \ti{n-gram}, \wordem{}, \gls{SRL}, dan \AMR{}.
Dari representasi-representasi tersebut, dapat dilakukan berbagai macam \ti{task} seperti peringkasan teks, klasifikasi sentimen, mesin translasi, \ti{question-answering}, deteksi parafrasa, dan lain-lain.

\Wordem{} merupakan representasi vektor berdimensi tinggi yang merepresentasikan kedudukan sebuah kata di antara kata-kata lain.
\Wordem{} dapat dibentuk dengan berbagai metode.
Ada metode yang tidak memberikan konteks pada kata seperti Word2Vec \citek{mikolov2013}, GloVe \citek{pennington2014}, dan FastText \citek{bojanowski2017}.
Metode tersebut menghasilkan yang disebut dengan \ti{non-contextual} \wordem{}.
Ada juga metode yang memberikan konteks pada kata dari sebuah kalimat, yang disebut dengan \ti{contextual} \wordem{}.
Dalam memahami konteks sebuah kata dari sebuah kalimat, diperlukan untuk memahami membaca keseluruhan kalimat tersebut.

\Gls{RNN}, \gls{LSTM}, dan \gls{GRU} kerap digunakan sebagai pendekatan sequence modeling dan permasalahan transduction seperti language modeling dan machine translation.
Pendekatan tersebut berguna untuk memahami konteks sebuah kata pada suatu kalimat.
Transformer merupakan sebuah sequence-to-sequence model yang terdiri dari \ti{encoder} dan \ti{decoder} \citek{vaswani2017}.
Transformer bergantung sepenuhnya terhadap mekanisme atensi untuk menggambarkan ketergantungan global antara input dan output.
Model berbasis transformer dapat berjalan secara paralel sehingga dapat mempercepat proses training.

Salah satu pengembangan \ti{encoder} dari model transformer adalah \gls{BERT}.
\gls{BERT} merupakan teknik machine learning berbasis transformer untuk \gls{NLP} yang dikembangkan oleh Google \citek{devlin2019}.
\gls{BERT} memiliki kemampuan untuk memahami konteks dalam sebuah kalimat dan menggunakannya untuk menghasilkan hasil yang lebih akurat daripada model pembelajaran mesin sebelumnya.
\gls{BERT} merupakan pengembangan komponen \ti{encoder} pada model transformer.
\gls{BERT} dilatih terhadap dua \ti{task} yaitu \ti{language modeling} dan \ti{next sentence prediction}.
Sebagai hasil dari proses pembelajaran ini adalah, \gls{BERT} mempelajari embedding kontekstual untuk kata-kata yang ada.

Kemudian dikembangkan \gls{BART} yang mampu melakukan denoising autoencoder yang memapping dokumen terkorupsi terhadap dokumen originalnya \citek{lewis2020}.
\gls{BART} diimplementasikan sebagai model berbasis transformer dengan bidirectional encoder dan left-to-right autoregressive decoder.
\gls{BART} memiliki arsitektur \ti{encoder} seperti \gls{BERT} \citek{devlin2019} dan \ti{decoder} seperti \gls{GPT2} \citek{radford2019}.
\gls{BART} \citek{lewis2020} adalah auto-encoder yang diimplementasikan sebagai model berbasis \gls{seq2seq} pada standar arsitektur Transformer \citek{vaswani2017}.
\gls{BART} di-train untuk merekonstruksi teks original berbasis teks terkorupsi yang digenerasi oleh 5 fungsi noising berikut ini.
\begin{enumerate}
  \item Token masking.
  Token secara random diubah menjadi elemen mask
  \item Token deletion.
  Token secara random dihapus dari input
  \item Text Infilling.
  Teks spans secara random diubah menjadi token single mask
  \item Sentence permutation.
  Teks dibagi menjadi segmen-segmen lalu diacak.
  \item Document Rotation.
  Dokumen dirotasi untuk dimulai dengan token random.
\end{enumerate}

% kecuali perbedaan-perbedaan berikut ini.
% \begin{enumerate}
%   \item Setiap layer dari decoder menampilkan cross-attention  pada layer tersembunyi terakhir dari encoder.
%   \item \gls{BERT} menggunakan tambahan feed-forward neural network sebelum memprediksi kata, \gls{BART} tidak.
% \end{enumerate}

Dalam menangani \ti{task} yang memerlukan lebih dari satu bahasa, seperti mesin translasi, digunakan representasi \mwordem{}.
Model \multil{} dapat dibangun dengan menambahkan \ti{vocabulary} dari bahasa lain dalam data \ti{training}.
\gls{BERT} juga dapat digunakan untuk menghasilkan \mwordem{}.
\textcite{devlin2019} melatih \gls{BERT} dengan menggunakan dataset yang terdiri dari 101 bahasa untuk menghasilkan \multil{} model bernama mBERT.
\gls{BART} juga dikembangkan untuk menghasilkan representasi \mwordem{} bernama mBART \citek{liu2020}.

Menurut \textcite{bi2020}, teknik \pretraining{} pada \gls{BERT} dan \gls{BART} kurang konsisten dengan tujuan beberapa \ti{task} generasi teks, seperti \ti{question answering}, peringkasan teks, generasi pertanyaan, dan generasi respon.
\textcite{bi2020} mengusulkan \gls{PALM} dengan adanya skema yang menggabungkan \pretraining{} \ti{auto-encoding} dan \ti{autoregressive} \ti{language model} pada korpus besar tak berlabel.
\gls{PALM} didesain untuk mengenerasi teks baru yang bergantung pada konteks.
Skema ini menangani ketidakcocokan antara teknik \pretraining{} dan \finetuning{} yang ada sebelumnya.
Namun, model \gls{PALM} untuk \mwordem{} belum pernah dikembangkan.
