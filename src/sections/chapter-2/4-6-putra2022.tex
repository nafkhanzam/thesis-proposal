\subsection{Pembangkitan Abstract Meaning Representation Lintas Bahasa dari Kalimat Berbahasa Indonesia \citek{putra2022}}

Penelitian ini melakukan \amrparsing{} dari kalimat berbahasa Indonesia dengan pendekatan cross lingual menggunakan dataset berkualitas silver dan gold.
Pada pembangkitan ini, dilakukan pemilihan kalimat berdasarkan kedekatan semantiknya menggunakan \textit{cosine similarity} sehingga kalimat dengan kinerja rendah dapat dikeluarkan terlebih dahulu dan mengurangi gap kinerja model.
Sumber daya yang digunakan dalam penelitian ini adalah \mwordem{} Numberbatch yang memiliki word embedding untuk bahasa Indo-Malay dan Inggris.
Untuk meningkatkan efisiensi dari \wordem{} tersebut, beberapa karakter Cina dan Arab yang tidak muncul dalam dataset AMR 2.0 dihapus.
Untuk sumber daya contextual \mwordem{}, digunakan beberapa alternatif berupa mBERT \citek{conneau2019}, XLM-R \citek{conneau2019}, dan mT5 \citek{xue2021}.

Gambaran keseluruhan sistem rancangan solusi dapat dilihat pada \cref{fig:4-6-solution-diagram}.
Tahapan yang dilakukan adalah sebagai berikut:
\begin{enumerate}
  \item Kontruksi korpus sebagai dataset yang dibutuhkan.
  \item Pelatihan model \amrparsing{} lintas bahasa.
  \item Inferensi kalimat berbahasa Indonesia menjadi graf \gls{AMR} berbahasa Inggris.
\end{enumerate}

\fig[0.7]{4-6-solution-diagram}{sections/chapter-2/4-6-solution-diagram.png}{Fig 1 \citek{putra2022}.}

Proses evaluasi pada penelitian ini dilakukan dengan pelaksanaan inferensi pada kalimat berbahasa Indonesia dari dataset gold untuk menghasilkan \gls{AMR} berbahasa Inggris.
Dari graf \gls{AMR} ini akan dilakukan perhitungan skor \gls{SMATCH} dengan membandingkannya dengan graf \gls{AMR} gold.
Konfigurasi model \amrparsing{} lintas bahasa yang memiliki skor \gls{SMATCH} terbaik adalah dengan \mwordem{} dari \gls{PLM} mT5 dengan skor \gls{SMATCH} 51.0.
Namun teknik \textit{baseline} \textit{translate-and-parse pipeline} masih memiliki skor \gls{SMATCH} yang lebih tinggi, yaitu 62.5.
