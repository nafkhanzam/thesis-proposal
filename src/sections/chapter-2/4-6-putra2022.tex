\subsection{Pembangkitan Abstract Meaning Representation Lintas Bahasa dari Kalimat Berbahasa Indonesia \citek{putra2022}}

Penelitian ini melakukan pembangkitan AMR berbahasa Indonesia dengan pendekatan cross lingual menggunakan dataset berkualitas silver dan gold.
Pada pembangkitan ini, dilakukan pemilihan kalimat berdasarkan kedekatan semantiknya menggunakan cosine similarity sehingga kalimat dengan kinerja rendah dapat dikeluarkan terlebih dahulu dan mengurangi gap kinerja model.
Sumber daya yang digunakan dalam penelitian ini adalah multilingual word embedding Numberbatch yang memiliki word embedding untuk bahasa Indo-malay dan Inggris dan berisikan beberapa karakter Cina dan Arab yang tidak muncul dalam dataset AMR 2.0.
Untuk sumber daya contextual multilingual word embedding, dilakukan alternatif berupa mBERT dan XLM-R dengan alasan mBERT belum dapat mengoptimalkan pretraining untuk multilingual model sehingga memerlukan pelatihan lebih lama.
Penggunaan XLM-R dan mT5 yang dilatih pada dataset Common Crawl dapat melihat kalimat-kalimat informal yang telah dilatih pada tokopedia.
Gambaran keseluruhan sistem rancangan solusi adalah sebagai berikut.

Proses evaluasi pada penelitian ini dilakukan dengan pelaksanaan inferensi pada kalimat bahasa Indonesia Gold Dataset untuk menghasilkan AMR berbahasa Inggris.
Dari graf AMR ini akan dilakukan perhitungan nilai smatch dibandingkan graf AMR gold standard menggunakan fine-grained metrics yang membandingkan AMR dengan beberapa konfigurasi.
Konfigurasi model pembangkit AMR Cross-lingual yang memiliki nilai smatch terbaik akan digunakan untuk menganalisis secara kuantitatif mengenai translation divergence dari kalimat berbahasa Indonesia ke graf AMR berbahasa Inggris seperti pada gambar berikut.

\fig[0.7]{4-6-fig1}{sections/chapter-2/4-6-fig1.png}{Fig 1 \citek{putra2022}.}
\fig[1]{4-6-fig2}{sections/chapter-2/4-6-fig2.png}{Fig 2 \citek{putra2022}.}
