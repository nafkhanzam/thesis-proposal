\subsection{Pembangkitan Abstract Meaning Representation Lintas Bahasa dari Kalimat Berbahasa Indonesia \citek{putra2022}}

Penelitian ini melakukan \amrparsing{} dari kalimat berbahasa Indonesia dengan pendekatan cross lingual menggunakan dataset berkualitas silver dan gold.
Pada pembangkitan ini, dilakukan pemilihan kalimat berdasarkan kedekatan semantiknya menggunakan \cossim{} sehingga kalimat dengan kinerja rendah dapat dikeluarkan terlebih dahulu dan mengurangi gap kinerja model.
Sumber daya yang digunakan dalam penelitian ini adalah \mwordem{} Numberbatch yang memiliki word embedding untuk bahasa Indo-Malay dan Inggris.
Untuk meningkatkan efisiensi dari \wordem{} tersebut, beberapa karakter Cina dan Arab yang tidak muncul dalam dataset AMR 2.0 dihapus.
Untuk sumber daya contextual \mwordem{}, digunakan beberapa alternatif berupa mBERT \citek{conneau2019}, XLM-R \citek{conneau2019}, dan mT5 \citek{xue2021}.

Gambaran keseluruhan sistem rancangan solusi dapat dilihat pada \cref{fig:4-6-solution-diagram}.
Tahapan yang dilakukan adalah sebagai berikut:
\begin{enumerate}
  \item Kontruksi korpus sebagai dataset yang dibutuhkan.
  \item Pelatihan model \amrparsing{} lintas bahasa.
  \item Inferensi kalimat berbahasa Indonesia menjadi graf \AMR{} berbahasa Inggris.
\end{enumerate}

\fig[0.7]{4-6-solution-diagram}{sections/chapter-2/4-6-solution-diagram.png}{Gambaran keseluruhan sistem rancangan solusi pembangunan model \crosslingual{} untuk Bahasa Indonesia \citek{putra2022}.}

Skema model pelatihan \crosslingual{} \amrparsing{} menggunakan skema yang diusulkan oleh \textcite{blloshmi2020}, yaitu \ti{zero-shot}, \ti{language-specific}, dan \ti{bilingual}.
\cref{fig:4-6-training-schemas} menunjukkan tiga skema pelatihan tersebut.
\ti{Zero-shot} hanya menggunakan kalimat berbahasa Inggris saja sebagai input model dan mengandalkan \mwordem{} untuk memahami konteks Bahasa Indonesia.
\ti{Language-specific} menggunakan kalimat berbahasa Indonesia saja sebagai input model.
\ti{Bilingual} menggunakan kalimat berbahasa Inggris dan Indonesia sebagai input model yang kemudian dievaluasi ke Bahasa Indonesia.

\fig{4-6-training-schemas}
  {sections/chapter-2/4-6-training-schemas.png}
  {Model pelatihan \crosslingual{} untuk Bahasa Indonesia: (a) \ti{zero-shot}, (b)\ti{language-specific}, dan (c) \ti{bilingual} \citek{putra2022}.}

Proses evaluasi pada penelitian ini dilakukan dengan pelaksanaan inferensi pada kalimat berbahasa Indonesia dari dataset gold untuk menghasilkan \AMR{} berbahasa Inggris.
Dari graf \AMR{} ini akan dilakukan perhitungan skor \SMATCH{} dengan membandingkannya dengan graf \AMR{} gold.
Konfigurasi model \amrparsing{} lintas bahasa yang memiliki skor \SMATCH{} terbaik adalah dengan \mwordem{} dari \gls{PLM} mT5 dengan skor \SMATCH{} 51.0.
Namun teknik \ti{baseline} \ti{translate-and-parse pipeline} masih memiliki skor \SMATCH{} yang lebih tinggi, yaitu 62.5.
