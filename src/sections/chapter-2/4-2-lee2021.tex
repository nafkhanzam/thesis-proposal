\subsection{\Glsfirst{MBSE} \citek{lee2021}}

Ensemble distillation (Hilton, 2015) mengintegrasikan pengetahuan dari perbedaan model pengajar ke model murid.
Untuk model sekuens ke sekuens seperti translasi mesin, dimungkinkan untuk membangun model dengan mengkombinasikan kemungkinan konteks kaya yang diberikan setiap langkah waktu (Kim and Rush, 2016).
Parser syntactic dan semantik memodelkan distribusi terhadap grad yang lebih sulit diintegrasikan antara model pengajar dalam cara yang optimal.
Pada kasus tertentu seperti dependency parsing, dimunhgkinkan untuk melakukan ensemble pengajar berdasar nosi dari edge attachment yang berhubungan dengan metric evaluasi Label Attachment Score (LAS).
Meski demikian, graf AMR cukup kompleks dan tidak diratakan kata per kata.
Standar metric Smatch (Cai, 2013) memperkirakan NP-Complete problem dari aligning nodes antara graf menggunakan hill climbing algorithm.
Hal ini mengilustrasikan kesulitan mencapai konsensus antara pengajar dalam ensembling AMR.

\fig[1]{4-2-fig2}{sections/chapter-2/4-2-fig1.png}{Fig 1 \citek{lee2021}.}
