\subsection{Teknik \textit{ensemble} untuk \amrparsing{}}

Kenaikan kinerja dari \amrparsing{} dari penelitian-penelitian sebelumnya tidak lagi meningkat secara signifikan.
Ini dikarenakan efek dari self-learning dan augmentasi silver data mulai berkurang.
Untuk mengatasi hal tersebut, \textcite{lee2022} mengusulkan untuk menggabungkan teknik \textit{ensemble} berbasis \gls{SMATCH} \citek{hoang2021} dengan \textit{ensemble distillation} untuk menghasilkan silver data berkualitas tinggi.

Teknik \textit{ensemble} yang diusulkan oleh \textcite{hoang2021} dapat dilihat pada \cref{fig:4-2-graph-ensemble-example}.
Ide utamanya adalah dengan memperbaiki sebuah pivot graf berdasarkan graf yang dihasilkan oleh \textit{parser} lain.
Dari pivot graf, dilakukan \textit{voting} untuk menghitung jumlah simpul dan sisi graf yang berkorelasi.
Setelah dilakukan \textit{voting}, dipilih simpul dan sisi yang memiliki jumlah voting terbesar sebagai hasil akhir graf tersebut.
Hal ini dilakukan berulang kali untuk masing-masing graf sebagai pivot awal.

\fig[1]{4-2-graph-ensemble-example}
  {sections/chapter-2/4-2-graph-ensemble-example.png}
  {Ilustrasi teknik \textit{ensemble} \citek{hoang2021}.}

\textcite{lee2022} mengusulkan sebuah framework untuk melakukan augmentasi data \gls{AMR}.
Ilustrasi framework tersebut dapat dilihat pada \cref{fig:4-2-silver-data-generation-framework}.
Dengan keterbatasan jumlah dataset yang dipunya untuk melatih model \textit{cross-lingual}, framework ini menghasilkan lebih banyak pasangan graf \gls{AMR} dan kalimat dengan bahasa yang dituju.
Teknik-teknik yang digunakan untuk menghasilkannya adalah sebagai berikut.
\begin{enumerate}
  \item Kalimat berbahasa Inggris dari dataset \gls{AMR} ditranslasi menjadi bahasa yang dituju.

  \item Graf \gls{AMR} dari dataset \gls{AMR} dilakukan \textit{parsing} menjadi kalimat Bahasa Inggris untuk mendapatkan sebuah alternatif kalimat berbahasa Inggris.
  Dari kalimat alternatif tersebut dilakukan translasi menjadi bahasa yang dituju.

  \item Kalimat-kalimat dari dataset kumpulan kalimat berbahasa Inggris tak berlabel ditranslasi menjadi bahasa yang dituju dan dilakukan \textit{parsing} menjadi graf \gls{AMR} menggunakan teknik \textit{ensemble}.
\end{enumerate}

\fig[1]{4-2-silver-data-generation-framework}
  {sections/chapter-2/4-2-silver-data-generation-framework.png}
  {Framework untuk melakukan augmentasi pasangan data \gls{AMR} dan kalimat berbahasa Inggris untuk menghasilkan silver data \citek{lee2022}.}
% Ensemble distillation (Hilton, 2015) mengintegrasikan pengetahuan dari perbedaan model pengajar ke model murid.
% Untuk model sekuens ke sekuens seperti mesin translasi, dimungkinkan untuk membangun model dengan mengkombinasikan kemungkinan konteks kaya yang diberikan setiap langkah waktu (Kim and Rush, 2016).
% Parser syntactic dan semantik memodelkan distribusi terhadap grad yang lebih sulit diintegrasikan antara model pengajar dalam cara yang optimal.
% Pada kasus tertentu seperti dependency parsing, dimunhgkinkan untuk melakukan ensemble pengajar berdasar nosi dari edge attachment yang berhubungan dengan metric evaluasi Label Attachment Score (LAS).
% Meski demikian, graf AMR cukup kompleks dan tidak diratakan kata per kata.
% Standar metric Smatch (Cai, 2013) memperkirakan NP-Complete problem dari aligning nodes antara graf menggunakan hill climbing algorithm.
% Hal ini mengilustrasikan kesulitan mencapai konsensus antara pengajar dalam ensembling AMR.

