\subsection{Teknik \ti{ensemble} untuk \amrparsing{}}

Kenaikan kinerja dari \amrparsing{} dari penelitian-penelitian sebelumnya tidak lagi meningkat secara signifikan \citek{lee2022}.
Ini dikarenakan efek dari self-learning dan augmentasi silver data mulai berkurang.
Untuk mengatasi hal tersebut, \textcite{lee2022} mengusulkan untuk menggabungkan teknik \ti{ensemble} berbasis \SMATCH{} \citek{hoang2021} dengan \ti{ensemble distillation} untuk menghasilkan silver data berkualitas tinggi.

Teknik \ti{ensemble} yang diusulkan oleh \textcite{hoang2021} bernama Algoritma Graphene dapat dilihat pada \cref{fig:4-2-graph-ensemble-example}.
Ide utamanya adalah dengan memperbaiki sebuah pivot graf berdasarkan graf yang dihasilkan oleh \ti{parser} lain.
Dari pivot graf, dilakukan \ti{voting} untuk menghitung jumlah simpul dan sisi graf yang berkorelasi.
Setelah dilakukan \ti{voting}, dipilih simpul dan sisi yang memiliki jumlah voting terbesar sebagai hasil akhir graf tersebut.
Hal ini dilakukan berulang kali untuk masing-masing graf sebagai pivot awal.
Tahapan algoritma ini dapat dilihat sebagai pseudo-code pada \cref{alg:4-2-ensemble}

\fig[1]{4-2-graph-ensemble-example}
  {sections/chapter-2/4-2-graph-ensemble-example.png}
  {Ilustrasi teknik \ti{ensemble} \citek{hoang2021}.}

% \alg{4-2-ensemble}{Algoritma Graphene untuk \ti{ensemble} graf \textcite{hoang2021}.}
% {
%   \textbf{Input:} a set of graphs $G = {g_1, g_2, ..., g_m}$ and the support threshold $\theta$
%   \textbf{Output:} an ensemble graph $g^e$
%   \textbf{Algorithm:} Graphene($G$,$\theta$)
%   \For{$i \gets 1$ \textbf{to} $m$}
%     \State $g_{pivot} \gets g_i$
%     \State $V \gets$ Initialise($g_{pivot}$)
%   \EndFor
% }
\begin{algorithm}
  \caption{Algoritma Graphene untuk \ti{ensemble} graf \citek{hoang2021}.}
  \label{alg:4-2-ensemble}
  \begin{algorithmic}
    \State \textbf{Input:} a set of graphs $G = \{g_1, g_2, ..., g_m\}$ and the support threshold $\theta$
    \State \textbf{Output:} an ensemble graph $g^e$
    \State \textbf{Algorithm:} Graphene($G$,$\theta$)
    \For{$i \gets 1$ \textbf{to} $m$}
      \State $g_{pivot} \gets g_i$
      \State $V \gets$ Initialise($g_{pivot}$)
      \For{$j \gets 1$ \textbf{to} $m$}
        \If{$j \neq i$}
          \State $V \gets V \cup$ getVote($\phi(g_{pivot},g_j)$)
        \EndIf
      \EndFor
      \State $g^e_i \gets Filter(V,\theta)$
    \EndFor
    \State $g^e \gets$ the graph with the largest support among $g^e_1,...,g^e_m$
    \State \Return $g^e$
  \end{algorithmic}
\end{algorithm}

\textcite{lee2022} mengusulkan sebuah framework untuk melakukan augmentasi data \AMR{}.
Ilustrasi framework tersebut dapat dilihat pada \cref{fig:4-2-silver-data-generation-framework}.
Dengan keterbatasan jumlah dataset yang dipunya untuk melatih model \crosslingual{}, framework ini menghasilkan lebih banyak pasangan graf \AMR{} dan kalimat dengan bahasa yang dituju.
Teknik-teknik yang digunakan untuk menghasilkannya adalah sebagai berikut:
\begin{enumerate}
  \item Kalimat berbahasa Inggris dari dataset \AMR{} ditranslasi menjadi bahasa yang dituju.

  \item Graf \AMR{} dari dataset \AMR{} dilakukan \ti{parsing} menjadi kalimat Bahasa Inggris untuk mendapatkan sebuah alternatif kalimat berbahasa Inggris.
  Dari kalimat alternatif tersebut dilakukan translasi menjadi bahasa yang dituju.

  \item Kalimat-kalimat dari dataset kumpulan kalimat berbahasa Inggris tak berlabel ditranslasi menjadi bahasa yang dituju dan dilakukan \ti{parsing} menjadi graf \AMR{} menggunakan teknik \ti{ensemble}.
\end{enumerate}

\fig[1]{4-2-silver-data-generation-framework}
  {sections/chapter-2/4-2-silver-data-generation-framework.png}
  {Framework untuk melakukan augmentasi pasangan data \AMR{} dan kalimat berbahasa Inggris untuk menghasilkan silver data \citek{lee2022}.}
