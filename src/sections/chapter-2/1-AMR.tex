\section{\glsfirst{AMR}}

\glsfirst{AMR} merupakan sebuah bahasa representasi semantik.
\gls{AMR} adalah graf berakar, berlabel, terarah, dan bersiklus (DAGs) yang terdiri dari satu kalimat utuh.
Tujuan dari \gls{AMR} ini adalah untuk mengabstraksi representasi sintesis dimana kalimat yang memiliki makna sama harus ditetapkan di \gls{AMR} yang sama meskipun mereka tidak dijabarkan dengan kata-kata yang sama.
Secara alami, bahasa \gls{AMR} bias terhadap bahasa Inggris dan tidak dimaksudkan untuk berfungsi sebagai bahasa bantu internasional \textcite{banarescu2013}.

\gls{AMR} memiliki tujuan untuk mengabstraksi perbedaan sintaksis untuk menghasilkan graf yang sama untuk kalimat bermakna serupa.
\gls{AMR} menggunakan framesets PropBank (Kingsbury dan Palmer, 2002) secara ekstensif untuk merepresentasikan predikat yang ada dalam kalimat.
Sebagai contoh, kalimat "\textit{the dog likes to eat}" memiliki dua buah predikat di dalamnya, yaitu likes dan eat.
Untuk menggunakan kedua predikat tersebut dalam graf \gls{AMR}, digunakan frame dari PropBank yang berkorespondensi dengan makna dari kedua predikat dalam kalimat.

Dalam pembentukan graf dari kalimat, \gls{AMR} tidak mempedulikan urutan dan langkah.
Anotasi yang ada pada \gls{AMR} berisikan aturan-aturan untuk merepresentasikan berbagai macam kata, frasa, dan kalimat.
Pada panduan tersebut tidak terdapat aturan eksplisit mengenai langkah pembentukan \gls{AMR} dari suatu kalimat dan bagaimana urutan yang harus diikuti.
Hal ini mendorong peneliti untuk berpikir secara fleksibel mengenai hubungan antara kalimat dan maknanya.

