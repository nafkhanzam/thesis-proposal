\section{\Glsfirst{AMR}}

\Glsfirst{AMR} merupakan sebuah bahasa representasi semantik.
\AMR{} adalah graf berakar, berlabel, terarah, dan bersiklus yang terdiri dari satu kalimat utuh.
Tujuan dari \AMR{} mengabstraksikan kalimat dari representasi sintaktiknya, di mana kalimat-kalimat yang memiliki pengertian yang sama seharusnya memiliki \AMR{} yang sama, walaupun tidak direpresentasikan dengan kalimat yang sama.
\AMR{} awalnya dibuat untuk kalimat berbahasa Inggris dan tidak digunakan sebagai bahasa secara umum \citek{banarescu2013}.

\AMR{} menggunakan \ti{framesets} PropBank \citek{kingsbury2002} secara ekstensif untuk merepresentasikan predikat yang ada dalam kalimat.
Sebagai contoh, kalimat \quotei{the dog likes to eat} memiliki dua buah predikat di dalamnya, yaitu \quotei{likes} dan \quotei{eat}.
Untuk menggunakan kedua predikat tersebut dalam graf \AMR{}, digunakan frame dari PropBank yang berkorespondensi dengan makna dari kedua predikat dalam kalimat.

Dalam pembentukan graf dari kalimat, \AMR{} tidak mempedulikan urutan dan langkah.
Anotasi yang ada pada \AMR{} berisikan aturan-aturan untuk merepresentasikan berbagai macam kata, frasa, dan kalimat.
Pada panduan tersebut tidak terdapat aturan eksplisit mengenai langkah pembentukan \AMR{} dari suatu kalimat dan bagaimana urutan yang harus diikuti.
Hal ini mendorong peneliti untuk berpikir secara fleksibel mengenai hubungan antara kalimat dan maknanya.

Sebuah \AMR{} dapat dituliskan dalam beberapa format sebagai berikut:
\begin{enumerate}
  \item Format graf.

  Karena \AMR{} berupa sebuah graf, maka sebuah graf \AMR{} dapat dituliskan dalam sebuah gambar graf.
  Contoh sebuah graf \AMR{} dari kalimat \quotei{the boy wants to go} dapat dilihat pada \cref{fig:amr-graph-example}.

  \fig{amr-graph-example}{sections/chapter-2/1-amr-graph-example.png}
    {Contoh sebuah graf \AMR{} dari kalimat \quotei{the boy wants to go} \citek{banarescu2013}.}

  \item Format PENMAN.

  Untuk penulisan secara linear supaya lebih mudah diproses secara komputasi, sebuah \AMR{} dapat dilinearisasi menjadi format PENMAN.
  Bila suatu entitas digunakan secara berulang, maka digunakan variabel sebagai referensi sebuah simpul pada graf yang dapat digunakan ulang.
  Contoh sebuah \AMR{} dari kalimat \quotei{the boy wants to go} dalam format PENMAN dapat dilihat pada \cref{fig:amr-penman-example}.

  \fig{amr-penman-example}{sections/chapter-2/1-amr-penman-example.png}
    {Contoh sebuah \AMR{} dari kalimat \quotei{the boy wants to go} dalam format PENMAN \citek{banarescu2013}.}

  \item Format logika.

  Format logika merepresentasikan keterhubungan antar simpul dan/atau sisi dari graf.
  Contoh sebuah \AMR{} dari kalimat \quotei{the boy wants to go} dalam format logika dapat dilihat pada \cref{fig:amr-logic-example}.

  \fig{amr-logic-example}
    {sections/chapter-2/1-amr-logic-example.png}
    {Contoh sebuah \AMR{} dari kalimat \quotei{the boy wants to go} dalam format logika \citek{banarescu2013}.}
\end{enumerate}
