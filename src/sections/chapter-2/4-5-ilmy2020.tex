\subsection{Pembangkitan Graf Abstract Meaning Representation Berbahasa Indonesia \citek{ilmy2020}}

Pada penelitian ini, dilakukan pembangkitan graf AMR berbahasa Indonesia menggunakan pembelajaran mesin.
Pendekatan yang digunakan adalah pendekatan menggunakan hasil dependency parser untuk mengetahui hubungan antar kata dan akar kalimat.
Dataset yang digunakan untuk melatih pembangkitan graf AMR masih terbatas dan hana terdiri dari kalimat sederhana untuk memudahkan penciptaan dataset baru.

Metode dalam penelitian ini melibatkan tiga tahap yaitu prediksi pasangan yang menghasilkan kandidat pasangan dengan konsep yang memiliki relasi dari input, prediksi label yang melakukan prediksi mengenai relasi kandidat pasangan, dan post-process dari kandidat pasangan berlabel untuk menghasilkan graf AMR yang valid.
