\subsection{AMR Parsing via Graph-sequence Iterative Inference \citek{cai2020}}

% \amrparsing{} merupakan suatu tugas mentransformasi teks natural language ke AMR.
Salah satu tantangan dalam \amrparsing{} adalah kurangnya pemetaan eksplisit antara simpul pada graf dan kata kata dalam teks.
Untuk saat ini, akurasi \textit{parsing} dari penelitian-penelitian terkait masih belum memuaskan dibandingkan kinerja manusia, terutama pada kasus dimana kalimat lebih panjang dan informatif.
Salah satu kemungkinan alasan dari kekurangan ini adalah defekasi turunan dari proses one-pass prediction, yang merupakan kekurangan kapabilitas modeling dari interaksi antara prediksi konsep dan prediksi relasi yang penting untuk mencapai keputusan yang tidak ambigu.

Pada tingkat dasar, kita dapat mengkategorikan pendekatan AMR parsing menjadi dua kelas.
Kedua kelas ini dijabarkan sebagai berikut.
\begin{enumerate}
  \item Two-stage parsing (Flanigan et al., 2014; Lyu and Titov, 2018; Zhang et al., 2019a) yang menggunakan desain pipeline untuk mengidentifikasi konsep dan hubungan prediksi dimana keputusan konsep mendahului keputusan relasi.

  \item One-stage parsing uang membangun graf parsing secara inkremental.
  Untuk analisis yang lebih dalam, metode one-stage parsing dapat dikategorikan menjadi tiga tipe.
  \begin{enumerate}
      \item Transition-based parsing yang memproses sebuah kalimat dari kiri ke kanan dan membangun graf secara bertahap dengan secara bergantian menyisipkan sebuah simpul baru atau membangun sisi baru.

      \item Seq2seq-based parsing yang memandang parsing sebagai transduksi sekuens ke sekuens oleh linearisasi dari graf AMR.

      \item Graph-based parsing di mana pada setiap langkah, simpul baru dan koneksinya terhadap simpul yang telah ada akan ditentukan baik secara berurut atau secara paralel.
  \end{enumerate}
\end{enumerate}

Sejauh ini, kausal timbal balik dan prediksi konsep masih belum dipelajari secara detail dan dimanfaatkan sepenuhnya.
TODO \cref{fig:4-1-fig1}.
Pendekatan AMR Parsing via Graph-sequence Iterative Inference mengikuti proses ketika expert manusia melakukan deduksi graf semantik dari suatu kalimat.
Output dari graf ini dimulai dari graf kosong yang memanjang secara inkremental dengan cara simpul ke simpul.
Pendekatan AMR parsing ini merupakan sebuah seri berisi sekuens graf ganda dengan pendekatan inferensi iteratif untuk pembuatan keputusan dan desain.
Ilustrasi sekuens graf ganda dengan inferensi iteratif ditampilkan pada \cref{fig:4-1-fig2}.

\fig{4-1-fig1}{sections/chapter-2/4-1-fig1.png}{Fig 1 \citek{cai2020}.}
\fig[1]{4-1-fig2}{sections/chapter-2/4-1-fig2.png}{Fig 2 \citek{cai2020}.}
