\subsection{\Glsfirst{SMATCH}}

Smatch merupakan singkatan dari Semantic Match yang merupakan metrics untuk mengukur overlap antar dua representasi semantik.
Tujuan utama dari semantik parsing adalah untuk menghasilkan hubungan relasi semantik dalam teks.
Hasil dari semantik parsing ini biasanya direpresentasikan oleh struktur semantik suatu kalimat utuh.
Evaluasi dari struktur ini diperlukan untuk tugas parsing semantik dan tugas anotasi semantik yang menghasilkan linguistic resource untuk semantic parsing.
Secara definisi, Smatch merupakan nilai f-score maksimal yang didapatkan berdasarkan padanan satu-satu antara dua AMR \citek{cai2013}.

Sebagai contoh, terdapat dua kalimat representasi AMR yaitu \quotei{the boy wants to go} dan \quotei{the boy wants the football}.
Perhitungan smatch dari kedua AMR ini memerlukan triplet dalam bentuk triplet berupa relasi(variabel, konsep) dan relasi(variabel1, variabel2).
Setelah itu dapat dihitung presisi, recall dan f-score dari overlap proposisi (triplet) kedua AMR tersebut.
Adapun triplet dari representasi AMR pertama dapat dilihat pada \cref{fig:first-triplet}.
Sedangkan bentuk triplet untuk representasi kedua dapat dilihat pada \cref{fig:second-triplet}.

\fig{first-triplet}
  {sections/chapter-2/1-1-first-triplet.png}
  {Triplet untuk AMR dari kalimat \quotei{the boy wants to go}.}

\fig{second-triplet}
  {sections/chapter-2/1-1-second-triplet.png}
  {Triplet untuk AMR dari kalimat \quotei{the boy wants the football}.}

Karena kedua AMR ini memiliki variabel yang berbeda dan tidak ada informasi mengenai hubungan antar variabel yang ada, perlu dicari semua kemungkinan dari pemetaan variabel yang ada.
Dari seluruh pemetaan, nilai Smatch akan mengambil hasil yang memiliki f-score tertinggi.

Untuk mengurangi waktu yang dibutuhkan dalam evaluasi tanpa menurunkan akurasi, dimanfaatkan metode hill-climbing.
Metode ini bekerja secara greedy dan tidak seoptimal Integer Linear Programming dan bruteforce, namun bisa menghasilkan perhitungan yang cukup baik.
Metode ini dilakukan dengan cara:
\begin{enumerate}
  \item Memetakan satu-satu antara m variable dari AMR pertama dan n variable dari AMR kedua.
  \item Menghitung jumlah proposisi yang dipetakan dan melakukan hill-climbing.
  \item Mencoba hal ini untuk m (n-1) neighbors pemetaan yang ada lalu dipilih pemetaan terbaik.
  \item Melakukan repetisi langkah 2-3 hingga tidak ada tetangga yang menghasilkan pemetaan lebih baik.
\end{enumerate}

Varian lain dalam evaluasi Smatch adalah \textit{finegrained} Smatch \citek{damonte2018}.
Metode ini membandingkan berbagai aspek dari graf yang dihasilkan untuk menilai kelebihan dan kekurangan yang ada.
Nilai-nilai evaluasi \textit{finegrained} yang bergantung pada Smatch adalah sebagai berikut.
\begin{enumerate}
  \item Unlabeled.
  Perhitungan Smatch dilakukan dengan kondisi semua label relasi yang ada pada grad AMR diubah menjadi label dummy sehingga evaluasi smatch terfokus pada struktur graf dan label simpul saja.
  \item No WSD.
  Semua simpul yang berupa fram propBank akan dihilangkan terlebih dahulu sense-nya sebelum proses perhitungan Smatch sehingga nilai ini menunjukkan Smatch bila tidak ada kesalahan pada Word Sense Disambiguation.
  \item Reentrancies.
  Pengambilan triplet yang variabelnya memiliki lebih dari satu parent lalu dilakukan perhitungan Smatch pada graf yang dihasilkan.
  \item SRL.
  Penilaian yang difokuskan pada identifikasi struktur predikat-argumen dengan cara menghitung Smatch hanya pada triplet yang mengandung relasi: ARGx dan triplet yang berhubungan dengannya.
\end{enumerate}
