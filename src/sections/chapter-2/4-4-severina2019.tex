\subsection{Peringkasan Abstraktif Multidokumen Menggunakan Abstract Meaning Representation untuk Bahasa Indonesia \citek{severina2019}}

Penelitian ini membahas mengenai pembangunan sistem peringkasan abstraktif dari dokumen berita berbahasa Indonesia memanfaatkan AMR.
Pendekatan yang digunakan adalah pendekatan berbasis aturan untuk membangkitkan graf AMR berbahasa Indonesia.
Adapun aturan-aturan pembangkitan graf adalah sebagai berikut.

\begin{enumerate}
  \item Simpul akar graf AMR yang dihasilkan ditentukan dengan dependency parser.
  \item ARG 0 dan ARG 1 dari suatu predikat ditentukan dengan melihat apakah predikat merupakan kata kerja aktif atau pasif.
  Argumen akan diisi dengan subjek dan objek dari kata kerja terkait.
  \item Keterangan dari predikat akan menjadi ARG 3 kecuali kata-kata pada frasa yang bersangkutan memiliki label khusus.
\end{enumerate}

Pendekatan dengan aturan-aturan ini memiliki beberapa keterbatasan.
Graf AMR yang dihasilkan dari pembangkitan ini hanya dapat memiliki simpul berupa frasa.
Selain itu, belum adanya PropBank berbahasa Indonesia menyebabkan predikat yang dihasilkan berupa kata dasar berlabel "-01".
Pada penelitian ini pengukuran yang digunakan adalah ukuran akurasi dari kemunculan simpul pada AMR dan hasilnya dibandingkan dengan referensi.
