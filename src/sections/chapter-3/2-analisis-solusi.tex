\section{Analisis Solusi}

Terdapat dua aspek dari permasalah dataset yang dapat diperbaiki dari penelitian \textcite{putra2022}, yaitu jumlah dan kualitas dataset.
Pada \amrparsing{} lintas bahasa, dataset silver diperlukan untuk membangun model \amrparsing{} lintas bahasa yang bisa didapatkan dari dataset gold yang telah ada dalam Bahasa Inggris.
\textcite{putra2022} telah membangun dataset silver untuk Bahasa Indonesia dari dataset AMR 2.0 dan korpus paralel BPPT-PANL \citek{bppt2009}.
Jumlah dataset silver dapat diperbanyak dengan cara-cara berikut:
\begin{enumerate}
  \item Menggunakan dataset AMR 3.0.
  Dataset AMR 3.0 memiliki sekitar 20,000 pasangan kalimat dan graf \AMR{} lebih banyak dibandingkan dengan Dataset AMR 2.0.

  \item Menggunakan korpus paralel tambahan IWSLT17.
  \textcite{putra2022} hanya menggunakan korpus paralel BPPT-PANL \citek{bppt2009}.
  Korpus paralel IWSLT17 \citek{cettolo2017} dapat menambah sampai 107,329 pasang dataset silver.

  \item Mengadopsi teknik augmentasi data oleh \textcite{lee2022}.
  Melakukan \ti{parsing} graf \AMR{} dari dataset gold menjadi teks berbahasa Inggris, lalu dilakukan translasi ke Bahasa Indonesia.
  Teknik ini dapat menambah jumlah pasangan dataset silver dari dataset AMR menjadi sampai maksimal dua kali lipat lebih banyak.
\end{enumerate}

Peningkatan kualitas dataset silver dapat dilakukan dengan menggunakan teknik \ti{ensemble} oleh \textcite{hoang2021}.
Berdasarkan hasil penelitian \textcite{lee2022}, teknik \ti{ensembling} 5 model \amrparser{} dapat meningkatkan kualitas dataset silver secara signifikan.

Kualitas dataset silver dapat diukur dengan menghitung \cossim{} dari \multil{} \ti{sentence embedding} dari pasangan kalimat Bahasa Indonesia dan Bahasa Inggris \citek{blloshmi2020}.
Model \mwordem{} yang digunakan untuk mendapatkan \multil{} \ti{sentence embedding} dari sebuah kalimat adalah \gls{LABSE} \citek{feng2022}.
Dalam meningkatkan kualitas dataset silver tersebut, dapat dilakukan dilakukan filtrasi untuk menghapus hasil translasi yang buruk.
Filtrasi dilakukan dengan mengaplikasikan algoritma \gls{1-NN} dengan \cossim{} sebagai nilai kedekatan antar kalimat berbahasa Inggris dan hasil translasinya.
Apabila hasil kalimat yang ditemukan dari \gls{1-NN} bukan merupakan kalimat awal sebelum translasi, maka pasangan kalimat tersebut tidak digunakan sebagai data pelatihan.

Teknik \pretraining{} graf oleh \textcite{bai2022} yang menggunakan \ti{language model} \gls{BART} terbukti dapat meningkatkan skor \SMATCH{} untuk \amrparsing{}.
Teknik tersebut belum dicoba untuk \crosslingual{} \amrparsing{}.
Skema model pelatihan \crosslingual{} \amrparsing{} telah diusulkan oleh \textcite{blloshmi2020}, yaitu \ti{zero-shot}, \ti{language-specific}, dan \ti{bilingual}.
Pada penelitian \textcite{putra2022}, ditunjukkan bahwa skema \ti{bilingual} menghasilkan kinerja terbaik.
Teknik \pretraining{} graf oleh \textcite{bai2022} tersebut dapat dilakukan dengan skema pelatihan \ti{bilingual} \citek{blloshmi2020} untuk membangun model \crosslingual{} untuk \ti{task} \amrparsing{}.

Skema \ti{bilingual} pada penelitian \textcite{blloshmi2020} dan \textcite{putra2022}, dataset silver berbahasa Inggris dan Indonesia dianggap sebagai \ti{instance} data yang berbeda.
Mengadopsi teknik \pretraining{} \gls{XLM}, kalimat berbahasa Inggris dan Indonesia dapat di-\ti{concatenate} menjadi satu \ti{instance} data.
