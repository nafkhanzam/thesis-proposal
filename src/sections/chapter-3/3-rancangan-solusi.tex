\section{Rancangan Solusi}

Pada masalah kurangnya jumlah dataset yang digunakan sebagai data latih model, dibangun kumpulan dataset silver untuk pelatihan model \amrparsing{} lintas bahasa.
Diagram alur keseluruhan proses konstruksi dataset gold dan silver dapat dilihat pada \cref{fig:3-dataset-construction}.
Alur proses konstruksi dataset berkualitas silver mengikuti framework augmentasi data oleh \textcite{lee2022}.

\fig[1]{3-dataset-construction}
  {sections/chapter-3/3-dataset-construction.png}
  {Diagram alur untuk konstruksi dataset gold dan silver dari korpus paralel dan dataset AMR.}

Korpus paralel PANL-BPPT dan IWSLT2017 digunakan sebagai dataset pasangan kalimat Bahasa Inggris dan Bahasa Indonesia.
Kalimat berbahasa Inggris dari korpus paralel diubah menjadi graf \gls{AMR}.
Model \amrparsing{} yang digunakan adalah model \sota{} saat ini, yaitu \glsfirst{MBSE}.
Hasil graf \gls{AMR} yang dihasilkan akan digunakan sebagai dataset latih berkualitas silver.

Dataset \gls{AMR} berkualitas gold yang digunakan AMR 3.0 dengan kode katalog LDC2020T02.
Augmentasi data untuk konstruksi dataset berkualitas silver mengambil bagian Train dan Dev \textit{split} dari AMR 3.0.
Kalimat berbahasa Inggris dari dataset \gls{AMR} tersebut diubah menjadi kalimat berbahasa Indonesia dengan menggunakan mesin translasi.
Model mesin translasi yang digunakan adalah model \sota{} saat ini, yaitu \todo{}.
Graf \gls{AMR} berkualitas gold juga dilakukan \textit{parsing} menjadi kalimat berbahasa Inggris (AMR-to-text) untuk menambah jumlah variasi pasangan dataset.
Kalimat berbahasa Inggris hasil \textit{parsing} tersebut juga diubah menjadi kalimat berbahasa Indonesia dengan menggunakan mesin translasi.
Hasil kumpulan kalimat yang dihasilkan akan digunakan sebagai dataset latih berkualitas silver.

Augmentasi data untuk konstruksi dataset berkualitas gold mengambil bagian Test \textit{split} dari AMR 3.0.
Bagian kalimat berbahasa Inggris dari Test \textit{split} ditranslasi menjadi Bahasa Indonesia oleh ahli.
Hasil kalimat berbahasa Indonesia tersebut berkualitas gold dan digunakan sebagai dataset test.
