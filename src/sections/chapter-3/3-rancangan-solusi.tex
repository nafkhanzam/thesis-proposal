\section{Rancangan Solusi}

% General Dataset Construction
Pada masalah kurangnya jumlah dataset yang digunakan sebagai data latih model, dibangun kumpulan dataset silver untuk pelatihan model \amrparsing{} lintas bahasa.
Diagram alur keseluruhan proses konstruksi dataset gold dan silver dapat dilihat pada \cref{fig:3-dataset-construction}.
Alur proses konstruksi dataset berkualitas silver mengikuti framework augmentasi data oleh \textcite{lee2022}.

\fig[1]{3-dataset-construction}
  {sections/chapter-3/3-dataset-construction.png}
  {Diagram alur untuk konstruksi dataset gold dan silver dari korpus paralel dan dataset AMR.}

% Dataset Silver Parallel Corpora
Korpus paralel PANL-BPPT dan IWSLT2017 digunakan sebagai dataset pasangan kalimat Bahasa Inggris dan Bahasa Indonesia.
Kalimat berbahasa Inggris dari korpus paralel diubah menjadi graf \AMR{}.
Model \amrparsing{} yang digunakan adalah model \sota{} saat ini, yaitu \gls{MBSE}.
Hasil graf \AMR{} yang dihasilkan akan digunakan sebagai dataset latih berkualitas silver.

% Dataset Silver AMR 3.0
Dataset \AMR{} berkualitas gold yang digunakan adalah Dataset AMR 3.0 (LDC2020T02).
Augmentasi data untuk konstruksi dataset berkualitas silver mengambil bagian Train dan Dev \ti{split} dari AMR 3.0.
Kalimat berbahasa Inggris dari dataset \AMR{} tersebut diubah menjadi kalimat berbahasa Indonesia dengan menggunakan mesin translasi.
Model mesin translasi yang digunakan adalah model {OPUS-MT} \citek{tiedemann2020}.
Graf \AMR{} berkualitas gold juga dilakukan \ti{parsing} menjadi kalimat berbahasa Inggris (AMR-to-text) untuk menambah jumlah variasi pasangan dataset.
Model AMR-to-text yang digunakan adalah \gls{AMRBART}.
Kalimat berbahasa Inggris hasil \ti{parsing} tersebut juga diubah menjadi kalimat berbahasa Indonesia dengan menggunakan mesin translasi.
Maka akan terdapat dua pasangan kalimat berbahasa Inggris dan Indonesia untuk satu buah graf \AMR{}.
Dua pasangan tersebut dilakukan \ti{cross join} untuk menghasilkan empat pasang data latih berkualitas silver.
Hasil kumpulan kalimat yang dihasilkan akan digunakan sebagai dataset latih berkualitas silver.

% Dataset Gold
Augmentasi data untuk konstruksi dataset berkualitas gold mengambil bagian Test \ti{split} dari AMR 3.0.
Bagian kalimat berbahasa Inggris dari Test \ti{split} ditranslasi menjadi Bahasa Indonesia oleh ahli.
Hasil kalimat berbahasa Indonesia tersebut berkualitas gold dan digunakan sebagai dataset test.

% Dataset Silver Evaluation Method
Kualitas dataset silver diukur dengan mengukur \cossim{} dari \multil{} \ti{sentence embedding} dari pasangan kalimat Bahasa Indonesia dan Bahasa Inggris.
Model \mwordem{} yang digunakan untuk mendapatkan \multil{} \ti{sentence embedding} dari sebuah kalimat adalah \gls{LABSE} \citek{feng2022}.

% Training Techniques
Skema pelatihan yang digunakan menggunakan skema \ti{zero-shot}, \ti{language-specific}, dan \ti{bilingual}.
Skema pelatihan yang digunakan hanya untuk yang melakukan \amrparsing{}.
Pada skema pelatihan \ti{bilingual}, diperlukan sedikit modifikasi untuk melakukan \pretraining{} graf pada teknik \gls{AMRBART}.
% Token \code{<s>} yang digunakan pada \gls{AMRBART} diganti menjadi kode bahasa yang digunakan.
% Bahasa Inggris ditandai dengan token \code{<en>} dan Bahasa Indonesia ditandai dengan token \code{<id>}.
Sesuai dengan skema pelatihan \ti{bilingual}, kalimat berbahasa Indonesia dan Inggris dilinearisasi secara sejajar sebagai input \pretraining{} maupun \finetuning{}.
Strategi \pretraining{} dan \finetuning{} untuk pelatihan graf model \amrparsing{} lintas bahasa tersebut dapat dilihat pada \cref{tab:3-pt-ft-strategies}.
Output pelatihan tetap berupa graf \AMR{} yang dilinearisasi, seperti \code{<g>g1,g2,...,gm</g>}.
Bentuk graf pada input maupun output berupa format PENMAN.

\tabl{3-pt-ft-strategies}
  {sections/chapter-3/3-pt-ft-strategies.csv}
  {
    Strategi \pretraining{} dan \finetuning{} untuk pelatihan graf model \amrparsing{} lintas bahasa pada skema pelatihan \ti{bilingual}.
    \code{a1,a2,...,an1} merupakan token untuk Bahasa Indonesia, sedangkan \code{b1,b2,...,bn2} merupakan token untuk Bahasa Inggris.
    \code{$t/g$} merupakan teks/graf \ti{original}.
    \code{$\hat{t}/\hat{g}$} merupakan teks/graf yang \ti{noisy} (hasil \denoising{}).
    \code{$\bar{t}/\bar{g}$} merupakan teks/graf yang hilang.
  }

Pelatihan \amrparsing{} akan dicoba pada beberapa \multil{} model.
Model-model yang akan dicoba adalah mBART, mT5, IndoBART, dan IndoT5.
Eksperimen \amrparsing{} juga akan dicoba dengan teknik \AMR{} ensembling \citek{hoang2021} dari hasil semua model-model tersebut.

% Evaluation
Evaluasi dilakukan secara kuantitatif dan kualitatif.
Metrik evaluasi yang digunakan untuk evaluasi kuantitatif adalah \SMATCH{}.
Evaluasi tersebut dilakukan dengan menghitung kemiripan graf \AMR{} hasil prediksi dan graf \AMR{} asli dari dataset test.
Metode kualitatif yang digunakan adalah evaluasi \transdiver{}.

% Baseline Evaluation
Pendekatan \ti{translate-and-parse} \citek{uhrig2021} digunakan sebagai baseline pada penelitian ini.
Model mesin translasi yang digunakan adalah {OPUS-MT} dan model \amrparsing{} yang digunakan adalah \gls{AMRBART}.
Tahapan yang dilakukan adalah melakukan translasi dari kalimat berbahasa Indonesia menjadi Bahasa Inggris, lalu dilakukan \amrparsing{} menjadi graf \AMR{}.
