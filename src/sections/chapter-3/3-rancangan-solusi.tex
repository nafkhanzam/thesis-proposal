\section{Rancangan Solusi}

% General Dataset Construction
Pada masalah kurangnya jumlah dataset yang digunakan sebagai data latih model, dibangun kumpulan dataset silver untuk pelatihan model \amrparsing{} lintas bahasa.
Diagram alur keseluruhan proses konstruksi dataset gold dan silver dapat dilihat pada \cref{fig:3-dataset-construction}.
Alur proses konstruksi dataset berkualitas silver mengikuti framework augmentasi data oleh \textcite{lee2022}.

\fig[1]{3-dataset-construction}
  {sections/chapter-3/3-dataset-construction.png}
  {Diagram alur untuk konstruksi dataset gold dan silver dari korpus paralel dan dataset AMR.}

% Dataset Silver Parallel Corpora
Korpus paralel PANL-BPPT dan IWSLT2017 digunakan sebagai dataset pasangan kalimat Bahasa Inggris dan Bahasa Indonesia.
Kalimat berbahasa Inggris dari korpus paralel diubah menjadi graf \AMR{}.
Model \amrparsing{} yang digunakan adalah model \sota{} saat ini, yaitu \glsfirst{MBSE}.
Hasil graf \AMR{} yang dihasilkan akan digunakan sebagai dataset latih berkualitas silver.

% Dataset Silver AMR 3.0
Dataset \AMR{} berkualitas gold yang digunakan adalah Dataset AMR 3.0 (LDC2020T02).
Augmentasi data untuk konstruksi dataset berkualitas silver mengambil bagian Train dan Dev \ti{split} dari AMR 3.0.
Kalimat berbahasa Inggris dari dataset \AMR{} tersebut diubah menjadi kalimat berbahasa Indonesia dengan menggunakan mesin translasi.
Model mesin translasi yang digunakan adalah model \sota{} saat ini, yaitu \todo[Pick a \sota{}]{}.
Graf \AMR{} berkualitas gold juga dilakukan \ti{parsing} menjadi kalimat berbahasa Inggris (AMR-to-text) untuk menambah jumlah variasi pasangan dataset.
Kalimat berbahasa Inggris hasil \ti{parsing} tersebut juga diubah menjadi kalimat berbahasa Indonesia dengan menggunakan mesin translasi.
Maka akan terdapat dua pasangan kalimat berbahasa Inggris dan Indonesia untuk satu buah graf \AMR{}.
Dua pasangan tersebut dilakukan \ti{cross join} untuk menghasilkan empat pasang data latih berkualitas silver.
Hasil kumpulan kalimat yang dihasilkan akan digunakan sebagai dataset latih berkualitas silver.

% Dataset Gold
Augmentasi data untuk konstruksi dataset berkualitas gold mengambil bagian Test \ti{split} dari AMR 3.0.
Bagian kalimat berbahasa Inggris dari Test \ti{split} ditranslasi menjadi Bahasa Indonesia oleh ahli.
Hasil kalimat berbahasa Indonesia tersebut berkualitas gold dan digunakan sebagai dataset test.

% Dataset Silver Evaluation Method
Kualitas dataset silver diukur dengan mengukur \cossim{} dari \multil{} \ti{sentence embedding} dari pasangan kalimat Bahasa Indonesia dan Bahasa Inggris.
Model \mwordem{} yang digunakan untuk mengdapatkan \multil{} \ti{sentence embedding} dari sebuah kalimat adalah \todo[Pick a \sota{}]{}.

% XL-AMR Techniques
Skema pelatihan yang digunakan menggunakan skema \ti{zero-shot}, \ti{language-specific}, dan \ti{bilingual}.
Skema pelatihan \ti{bilingual} memerlukan sedikit modifikasi untuk melakukan \pretraining{} graf pada teknik \gls{AMRBART}.

